\documentclass{article}
\usepackage{graphicx} 
\usepackage{amsfonts} 
\usepackage{amsmath}
\usepackage{multirow}
\usepackage{tikz}
\usepackage{polski}
\usepackage{mathtools}
\usepackage{hyperref}
\usepackage{breqn}

\title{ Wskaźnik planarności w grafach Barabasi Albert }

\author{}

\date{}

\begin{document}

\maketitle

\section{Cel}
Chcemy oceniać szansę na wystąpienie podpodziału $K_{3,3}$ lub $K_5$ w grafie $G^m_n$.

\section{Założenia}
\begin{enumerate}
    \item Dowolne ścieżki nie powinny wpływać na wskaźnik, ponieważ mogą zostać zkontrakowane do jednego wierzchołka.
    \item Wskaźnik powinien uwzględniać to, że każda krawędź może zostać użyta tylko raz.
\end{enumerate}

\section{Realizacja}
\begin{dmath}
    \tau(G) = \sum_{u \neq v \in V(G)} \frac{1}{d(u) \cdot d(v)}
\end{dmath}

\section{Usprawnienia obliczniowe}

\end{document}
