\documentclass{article}
\usepackage{graphicx} 
\usepackage{amsfonts} 
\usepackage{amsmath}
\usepackage{multirow}
\usepackage{tikz}
\usepackage{polski}
\usepackage{mathtools}
\usepackage{hyperref}
\usepackage{breqn}

\title{ Wskaźnik planarności w grafach Barabasi Albert }

\author{}

\date{}

\begin{document}

\maketitle

\section{Cel}
Chcemy oceniać szansę na wystąpienie podpodziału $K_{3,3}$ lub $K_5$ w grafie $G^m_n$.
Zaczynamy od pustego grafu. W każdym kroku dodajemy nową krawędź/ nowy wierzchołek z krawędzią.
Co każde dodanie krawędzi na podstawie indeksu planaraności decydujemy jaki kolor jej nadać. Każdy kolor odpowiada jednej z warstw na które rozkładamy graf.
Indeks ten liczymy dla każdej warstwy osobno.

\section{Założenia}
\begin{enumerate}
    \item Dowolne ścieżki nie powinny wpływać na wskaźnik, ponieważ mogą zostać zkontrakowane.
    \item Wskaźnik powinien uwzględniać to, że każda krawędź może zostać użyta tylko raz.
    \item Trzeba uwzględnić to aby wierzchołki tworzyły klikę czy graf pełny dwudzielny.
\end{enumerate}

\section{Realizacja}
Weźmy pewną ścieżką $P$ w grafie: $p_1, \ldots, p_k$ i zdefiniujmy funkcję:
\begin{dmath}
    \tau(P) = \phi(p_1) \cdot (\prod_{i=2}^{k-1} \psi(p_i)) \cdot \phi(p_k)
\end{dmath}
gdzie: $\phi(p) = \frac{MIN\{d, \\\ deg(p)\}}{d}$ oraz $\psi(p) = \frac{1}{deg(p) - 1}$.
\newline
Tak zdefiniowana funkcja $\tau(P)$ zwraca wartości z przedziału $(0, 1]$. Wartość 1 oznacza, że ścieżka $P$ może zostać zkontraktowana do krawędzi łączącej dwa wierzchołki stopnia $d$ oraz wierzchołki w niej występujące nie wystąpią w innym podgrafie.
\newline
Sumując wartości $\tau(P)$ dla wszystkich ścieżek pomiędzy dwoma wierzchołkami otrzymamy 'oczekiwaną liczbę ścieżek', które mogą zostać skontraktowane do krawędzi łączącej dwa wierzchołkami stopnia o 'oczekwianym stopniu $d$'.
\newline
Aby określić wartość $\tau$ dla całego grafu sumujemy mnożymy $\tau$ dla ścieżek pomiędzy wierzchołkami tworzącymi potencjalne kliki pełne grafy dwudzielne.

\section{Usprawnienia obliczeniowe}

\end{document}
